This manuscript considers the following "graph classification" question: given a collection of graphs and associated classes, how can one predict the class of a newly observed graph?  To address this question we propose a statistical model for graph/class pairs.  This model naturally leads to a set of estimators to identify the class-conditional signal, or "signal-subgraph," defined as the collection of edges that are probabilistically different between the classes. The estimators admit classifiers which are asymptotically optimal and efficient, but differ by their assumption about the "coherency" of the signal-subgraph (coherency is the extent to which the signal-edges "stick together" around a common subset of vertices). Via simulation, the best estimator is shown to be not just a function of the coherency of the model, but also the number of training samples.  These estimators are employed to address a contemporary neuroscience question: can we classify "connectomes" (brain-graphs) according to sex?  The answer is yes, and significantly better than all benchmark algorithms considered.  Synthetic data analysis demonstrates that even when the model is correct, given the relatively small number of training samples, the estimated signal-subgraph should be taken with a grain of salt.  We conclude by discussing several possible extensions.
